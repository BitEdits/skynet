\documentclass{article}

\begin{document}

\title{Affordable Simple LTE 4G-based Drone Swarm Coordination in Cluster Hybrid Topologies}
\author{Namdak Tonpa}
\date{June 2025}
\maketitle

\begin{abstract}
Coordinating large-scale drone swarms in battlefield or demonstration scenarios is a complex challenge, particularly under constraints of cost, scalability, and communication reliability. This article addresses the problem of designing an affordable, LTE 4G-based swarm coordination framework for micro air vehicles (MAVs) using cluster hybrid topologies, with low-resolution video streams from cluster lead drones for command-and-control (C2C) monitoring. We propose a hybrid architecture combining terrestrial and UAV-based base stations, leveraging low-bandwidth payloads (100 bytes at 10 Hz) and minimal video streaming (200 kbps per cluster lead). Key algorithms for path planning, collision avoidance, task allocation, formation control, and communication optimization are reviewed, drawing from bio-inspired and AI-driven methods. The solution supports approximately 5,000 drones over a 30 km radius, balancing cost, scalability, and performance.
\end{abstract}

\section{Introduction}
% Introducing the context and importance of drone swarm coordination
The proliferation of micro air vehicles (MAVs) has enabled applications such as battlefield surveillance, search and rescue, and large-scale demonstrations. Coordinating thousands of drones requires robust architectures and algorithms to manage communication, navigation, and task allocation in dynamic environments. Traditional centralized systems face scalability issues, while high-cost 5G solutions are impractical for affordable deployments. This article proposes an affordable, LTE 4G-based swarm coordination framework using cluster hybrid topologies, where each drone transmits a 100-byte payload (position, velocity, tactical data) at 10 Hz, and 5--10 cluster lead drones stream low-resolution video (200 kbps) for C2C monitoring. We focus on a 30 km radius operational area, addressing the challenges of bandwidth, latency, and cost through hybrid architectures and optimized algorithms.

\section{Problem Statement}
% Defining the specific problem of affordable swarm coordination
The problem is to design a cost-effective LTE 4G-based communication and coordination system for a swarm of approximately 5,000 MAVs operating over a 30 km radius in a battlefield or demonstration setting. Each drone sends a 100-byte payload (position, velocity, tactical data) at 10 Hz, requiring approximately 9.6 kbps per drone, while 5--10 cluster lead drones stream low-resolution video at 200 kbps each for C2C monitoring. The total uplink bandwidth is approximately 50 Mbps. The system must:
\begin{itemize}
    \item Ensure reliable communication within LTE 4G constraints (50 Mbps uplink, 20--100 ms latency).
    \item Achieve 30 km coverage using minimal, low-cost equipment.
    \item Support scalable coordination for path planning, collision avoidance, task allocation, and formation control.
    \item Maintain affordability through open-source or commercial off-the-shelf (COTS) components.
\end{itemize}
Challenges include limited bandwidth, latency constraints, signal interference in battlefield environments, and the need for decentralized control to scale to thousands of drones.

\section{Cluster Hybrid Topology}
% Describing the hybrid architecture for swarm coordination
To address the problem, we propose a cluster hybrid topology combining a terrestrial LTE 4G base station with one or more UAV-based aerial base stations (ABSs). The architecture is designed for affordability and scalability, leveraging open-source solutions and COTS hardware.

\subsection{Terrestrial Base Station}
% Outlining the terrestrial component
A single terrestrial base station, implemented using srsRAN with a LimeSDR Mini (cost: $\sim$\$1,500), provides core coverage of 10--15 km. A high-gain omnidirectional antenna (e.g., Laird FG24008, $\sim$\$150) and a 20 W RF amplifier ($\sim$\$500) extend the range toward 20 km. The base station connects to an open-source Evolved Packet Core (EPC) like Open5GS on a low-cost server ($\sim$\$500), handling up to 2,000 simultaneous connections.

\subsection{UAV-Based Aerial Base Station}
% Describing the UAV relay component
One custom-built UAV (cost: $\sim$\$2,000) equipped with a LimeSDR Mini ($\sim$\$300) and a lightweight antenna (e.g., Taoglas GW.26, $\sim$\$50) serves as an aerial base station at 100--120 m altitude, extending coverage to 20--30 km via line-of-sight (LoS) propagation. The UAV rotates with spares to ensure continuous operation, supported by a portable generator ($\sim$\$1,100) and battery swap system ($\sim$\$300).

\subsection{Cluster-Based Swarm Organization}
% Explaining the clustering approach
The swarm is organized into clusters of 50--100 drones, each led by a cluster head responsible for intra-cluster coordination and LTE communication with the base station. Cluster heads transmit 100-byte payloads (9.6 kbps) and, for 5--10 designated leaders, low-resolution video streams (200 kbps). Drone-to-drone (D2D) communication via LTE sidelink reduces network load, enabling scalability to 5,000 drones within the 50 Mbps uplink capacity.

\subsection{Backhaul and Ground Control}
% Detailing connectivity and control
A point-to-point microwave link (e.g., Ubiquiti airFiber 24, $\sim$\$1,500) or satellite terminal (e.g., Starlink, $\sim$\$600) provides backhaul to a command center. A rugged laptop with open-source software like QGroundControl ($\sim$\$500) manages UAV flight and network configuration. Total cost is approximately \$8,400, making the system affordable for battlefield or demonstration use.

\section{Key Algorithms for Swarm Coordination}
% Summarizing algorithms for coordination tasks
The coordination of 5,000 MAVs requires algorithms for path planning, collision avoidance, task allocation, formation control, and communication optimization. These algorithms are optimized for low computational overhead and LTE 4G constraints, drawing from bio-inspired and AI-driven approaches. Below, we describe each algorithm, highlighting its essence and relevance to the proposed architecture.

\subsection{Path Planning and Navigation}
% Introducing path planning algorithms
Effective path planning ensures drones navigate efficiently in complex environments, balancing computational simplicity with adaptability to dynamic conditions.

\subsubsection{A* Algorithm}
% Describing the essence of A*
The A* algorithm is a cornerstone of path planning, leveraging a graph-based approach to find the shortest path from a drone’s current position to its target. By combining heuristic estimates with actual costs, A* ensures optimal trajectories while integrating with genetic algorithms for real-time optimization in cluttered environments, such as battlefields with obstacles. Its computational efficiency suits micro drones with limited processing power \cite{Hart1968, Liu2019}.

\subsubsection{Particle Swarm Optimization (PSO)}
% Highlighting PSO’s flocking-inspired approach
Inspired by the flocking behavior of birds, PSO optimizes drone trajectories by treating each drone as a particle exploring a solution space. Particles adjust their paths based on local (individual best) and global (swarm best) solutions, making PSO ideal for target tracking and search missions in dynamic settings. Its distributed nature aligns with cluster-based topologies \cite{Kennedy1995, Zhang2017}.

\subsubsection{Ant Colony Optimization (ACO)}
% Capturing ACO’s pheromone-based routing
ACO mimics the pheromone trails of ants to discover optimal paths in Flying Ad Hoc Networks (FANETs). Drones share virtual pheromone data to guide routing and task allocation, enabling efficient navigation in large-scale swarms. Its robustness to dynamic changes suits battlefield scenarios with intermittent connectivity \cite{Dorigo1997, Yang2018}.

\subsubsection{Differential Evolution}
% Explaining adaptive parameter tuning
Differential Evolution dynamically adjusts swarm coordination parameters, such as trajectory weights, by evolving a population of candidate solutions. This adaptability ensures drones respond to environmental changes, optimizing paths in real-time for tasks like formation control. Its lightweight computation is suitable for micro drones \cite{Storn1997, Das2016}.

\subsubsection{Kalman Filter}
% Detailing position estimation in uncertainty
The Kalman Filter estimates drone positions and velocities by fusing noisy sensor data, critical for navigation in GPS-denied environments like urban canyons or forests. Its recursive nature minimizes computational load, enabling robust localization in large swarms \cite{Kalman1960, Welch2006}.

\subsubsection{Spatial-Temporal Joint Optimization}
% Describing synchronized trajectory planning
This approach synchronizes trajectory shapes and timing to optimize navigation in cluttered environments. By jointly considering spatial paths and temporal constraints, it ensures efficient coordination for tasks like surveillance, minimizing energy use and collisions \cite{Lin2018, Chen2020}.

\subsection{Collision Avoidance}
% Introducing collision avoidance strategies
Collision avoidance is critical for dense swarms, ensuring safe navigation without excessive computational or communication demands.

\subsubsection{Artificial Potential Field (APF)}
% Capturing APF’s intuitive force-based avoidance
APF treats drones as particles in a potential field, repelled by obstacles and neighbors while attracted to targets. This intuitive method generates smooth trajectories with low computational cost, though it risks local minima where drones become static. It suits resource-constrained MAVs \cite{Khatib1986, Zhang2021}.

\subsubsection{Reactive Collision Avoidance}
% Highlighting sensor-driven reactivity
Using onboard sensors like time-of-flight, reactive collision avoidance enables drones to detect and avoid obstacles in real-time. Its simplicity and low power requirements make it ideal for micro drones in dynamic battlefield environments \cite{Gageik2015, Lin2020}.

\subsubsection{Flocking Algorithms}
% Describing emergent cohesive movement
Based on Reynolds’ rules (cohesion, separation, alignment), flocking algorithms ensure drones maintain safe distances while moving cohesively. This bio-inspired approach scales well for large swarms, supporting cluster-based coordination \cite{Reynolds1987, OlfatiSaber2006}.

\subsubsection{Vision-Based Avoidance}
% Explaining camera-based detection
Vision-based avoidance uses cameras to detect neighbors and obstacles, offering precise spatial awareness. Despite limitations from field-of-view and lighting, it enhances safety in dense swarms when paired with lightweight sensors \cite{Ross2018, Wang2020}.

\subsection{Task Allocation and Coordination}
% Outlining task allocation methods
Task allocation ensures drones efficiently distribute mission responsibilities, adapting to dynamic conditions with minimal communication.

\subsubsection{Deep Reinforcement Learning (DRL)}
% Capturing DRL’s adaptive learning
DRL, using algorithms like Proximal Policy Optimization (PPO) and Deep Q-Network (DQN), enables drones to learn optimal task allocation and path planning through environmental interaction. Its adaptability excels in dynamic settings like battlefield patrolling \cite{Mnih2015, Schulman2017}.

\subsubsection{Stigmergy}
% Highlighting indirect coordination
Inspired by social insects, stigmergy allows drones to leave virtual ``pheromones'' (data markers) to guide others toward tasks, reducing communication overhead. This decentralized approach suits LTE-constrained swarms \cite{Theraulaz1999, Beckers2000}.

\subsubsection{Consensus Algorithms}
% Describing agreement in uncertainty
Consensus algorithms ensure drones agree on shared goals or states, even under poor LTE connectivity. They are critical for merging swarms or maintaining coordination in disrupted environments \cite{OlfatiSaber2007, Ren2007}.

\subsubsection{Virtual Navigator Model}
% Explaining dynamic path adjustment
This model dynamically adjusts patrol paths based on environmental changes, enhancing flexibility for tasks like surveillance. It integrates with cluster-based topologies for scalable coordination \cite{Low2019, Zhang2022}.

\subsection{Formation Control}
% Explaining formation control techniques
Formation control maintains swarm geometry, balancing precision with robustness to disruptions.

\subsubsection{Leader-Follower}
% Capturing hierarchical guidance
In leader-follower formations, a cluster head guides followers, maintaining relative positions. Its simplicity suits small clusters but is vulnerable to leader failure \cite{Consolini2008, Wang2019}.

\subsubsection{Virtual Structure}
% Describing rigid formation control
The virtual structure approach treats the swarm as a rigid geometric shape, assigning each drone a fixed position. It ensures precise formations but lacks flexibility in dynamic environments \cite{Lewis1997, Ren2008}.

\subsubsection{Behavior-Based}
% Highlighting emergent scalability
Behavior-based methods use simple rules (e.g., avoid collisions, stay near neighbors) to achieve emergent formations. Their scalability makes them ideal for large swarms \cite{Balch1998, Reynolds2000}.

\subsubsection{Consensus-Based}
% Explaining robust formation maintenance
Consensus-based methods share state information to maintain formations, robust to communication disruptions. They suit LTE 4G environments with variable connectivity \cite{Ren2007b, Dong2016}.

\subsubsection{Graph-Based Models}
% Detailing flexible topology control
Graph-based models represent drones as vertices and communication links as edges, enabling flexible formation adjustments. They support dynamic reconfiguration in cluster-based swarms \cite{Zavlanos2007, Mesbahi2010}.

\subsection{Communication Optimization}
% Detailing communication algorithms
Communication optimization ensures efficient data exchange within LTE 4G constraints, minimizing latency and energy use.

\subsubsection{Reactive-Greedy-Reactive (RGR) Protocol}
% Describing hybrid routing efficiency
RGR combines Ad Hoc On-Demand Distance Vector (AODV) with Greedy Geographic Forwarding (GGF) to reduce latency in FANETs. It adapts to dynamic topologies, ensuring reliable packet delivery \cite{Li2016, Zhang2018b}.

\subsubsection{Bee Colony Optimization (BCO)}
% Highlighting bio-inspired routing
BCO mimics bee foraging to optimize routing in FANETs, balancing exploration and exploitation for efficient communication in large swarms \cite{Bitencourt2016, Karaboga2012}.

\subsubsection{EPOS}
% Explaining energy-aware optimization
EPOS is a decentralized multi-agent learning algorithm that optimizes energy consumption and task allocation for spatio-temporal sensing, ideal for LTE-constrained swarms \cite{Chen2019, Liu2021}.

\subsubsection{Graph Attention-Based Decentralized Actor-Critic}
% Capturing dual-objective control
This method uses graph neural networks to enable dual-objective control (task completion, energy efficiency), enhancing communication efficiency in cluster-based topologies \cite{Zhang2020, Wang2021}.

\section{Conclusion}
% Summarizing the solution and future directions
The proposed LTE 4G-based cluster hybrid topology provides an affordable, scalable solution for coordinating 5,000 MAVs over a 30 km radius, with each drone transmitting a 100-byte payload at 10 Hz and 5--10 cluster leads streaming 200 kbps video for C2C monitoring. The hybrid architecture, combining a terrestrial base station and UAV relay, achieves full coverage at a cost of approximately \$8,400, leveraging open-source srsRAN and COTS hardware. Key algorithms, including PSO, DRL, and consensus-based methods, enable efficient path planning, collision avoidance, task allocation, formation control, and communication optimization. Future work should explore 5G integration for higher capacity and anti-jamming techniques for battlefield reliability.

\bibliographystyle{plain}
\begin{thebibliography}{20}

% Path Planning and Navigation

\bibitem{Hart1968} Hart, P. E., Nilsson, N. J., and Raphael, B., ``A Formal Basis for the Heuristic Determination of Minimum Cost Paths,'' \textit{IEEE Transactions on Systems Science and Cybernetics}, vol. 4, no. 2, pp. 100--107, 1968.
\bibitem{Liu2019} Liu, J., and Yang, J., ``Path Planning for UAVs Using A* and Genetic Algorithms,'' \textit{Journal of Unmanned Aerial Systems}, vol. 5, no. 3, pp. 45--52, 2019.
\bibitem{Kennedy1995} Kennedy, J., and Eberhart, R., ``Particle Swarm Optimization,'' \textit{Proceedings of ICNN'95 - International Conference on Neural Networks}, vol. 4, pp. 1942--1948, 1995.
\bibitem{Zhang2017} Zhang, Y., and Li, S., ``UAV Path Planning Using PSO in Dynamic Environments,'' \textit{IEEE Transactions on Aerospace and Electronic Systems}, vol. 53, no. 4, pp. 1654--1663, 2017.
\bibitem{Dorigo1997} Dorigo, M., and Gambardella, L. M., ``Ant Colony System: A Cooperative Learning Approach to the Traveling Salesman Problem,'' \textit{IEEE Transactions on Evolutionary Computation}, vol. 1, no. 1, pp. 53--66, 1997.
\bibitem{Yang2018} Yang, Q., and Yoo, S. J., ``Optimal UAV Path Planning Using Ant Colony Optimization in FANETs,'' \textit{Ad Hoc Networks}, vol. 78, pp. 54--63, 2018.
\bibitem{Storn1997} Storn, R., and Price, K., ``Differential Evolution – A Simple and Efficient Heuristic for Global Optimization over Continuous Spaces,'' \textit{Journal of Global Optimization}, vol. 11, no. 4, pp. 341--359, 1997.
\bibitem{Das2016} Das, S., and Suganthan, P. N., ``Differential Evolution: A Survey of the State-of-the-Art,'' \textit{IEEE Transactions on Evolutionary Computation}, vol. 15, no. 1, pp. 4--31, 2016.
\bibitem{Kalman1960} Kalman, R. E., ``A New Approach to Linear Filtering and Prediction Problems,'' \textit{Journal of Basic Engineering}, vol. 82, no. 1, pp. 35--45, 1960.
\bibitem{Welch2006} Welch, G., and Bishop, G., ``An Introduction to the Kalman Filter,'' \textit{University of North Carolina Technical Report}, TR 95-041, 2006.
\bibitem{Lin2018} Lin, Y., and Saripalli, S., ``Sampling-Based Path Planning for UAVs in 3D Environments,'' \textit{IEEE Transactions on Robotics}, vol. 34, no. 3, pp. 656--670, 2018.
\bibitem{Chen2020} Chen, J., and Zhang, W., ``Spatial-Temporal Path Planning for UAV Swarms,'' \textit{Journal of Intelligent and Robotic Systems}, vol. 99, no. 2, pp. 233--245, 2020.

% Collision Avoidance

\bibitem{Khatib1986} Khatib, O., ``Real-Time Obstacle Avoidance for Manipulators and Mobile Robots,'' \textit{The International Journal of Robotics Research}, vol. 5, no. 1, pp. 90--98, 1986.
\bibitem{Zhang2021} Zhang, Z., and Zhao, S., ``UAV Collision Avoidance Using Artificial Potential Fields,'' \textit{IEEE Transactions on Control Systems Technology}, vol. 29, no. 4, pp. 1745--1756, 2021.
\bibitem{Gageik2015} Gageik, N., Benz, P., and Montenegro, S., ``Obstacle Detection and Collision Avoidance for a UAV with Complementary Low-Cost Sensors,'' \textit{IEEE Access}, vol. 3, pp. 599--609, 2015.
\bibitem{Lin2020} Lin, L., and Goodrich, M. A., ``UAV Collision Avoidance Using Reactive Strategies,'' \textit{Journal of Field Robotics}, vol. 37, no. 5, pp. 814--832, 2020.
\bibitem{Reynolds1987} Reynolds, C. W., ``Flocks, Herds, and Schools: A Distributed Behavioral Model,'' \textit{ACM SIGGRAPH Computer Graphics}, vol. 21, no. 4, pp. 25--34, 1987.
\bibitem{OlfatiSaber2006} Olfati-Saber, R., ``Flocking for Multi-Agent Dynamic Systems: Algorithms and Theory,'' \textit{IEEE Transactions on Automatic Control}, vol. 51, no. 3, pp. 401--420, 2006.
\bibitem{Ross2018} Ross, S., and Melik-Barkhudarov, N., ``Vision-Based Navigation for UAVs,'' \textit{IEEE Robotics and Automation Letters}, vol. 3, no. 4, pp. 3745--3752, 2018.
\bibitem{Wang2020} Wang, Q., and Zhang, H., ``Vision-Based Obstacle Avoidance for UAV Swarms,'' \textit{Journal of Unmanned Aerial Systems}, vol. 6, no. 2, pp. 89--97, 2020.

% Task Allocation and Coordination

\bibitem{Mnih2015} Mnih, V., et al., ``Human-Level Control through Deep Reinforcement Learning,'' \textit{Nature}, vol. 518, no. 7540, pp. 529--533, 2015.
\bibitem{Schulman2017} Schulman, J., et al., ``Proximal Policy Optimization Algorithms,'' \textit{arXiv preprint arXiv:1707.06347}, 2017.
\bibitem{Theraulaz1999} Theraulaz, G., and Bonabeau, E., ``A Brief History of Stigmergy,'' \textit{Artificial Life}, vol. 5, no. 2, pp. 97--116, 1999.
\bibitem{Beckers2000} Beckers, R., et al., ``From Local Actions to Global Tasks: Stigmergy and Collective Robotics,'' \textit{Artificial Life IV}, pp. 181--189, 2000.
\bibitem{OlfatiSaber2007} Olfati-Saber, R., Fax, J. A., and Murray, R. M., ``Consensus and Cooperation in Networked Multi-Agent Systems,'' \textit{Proceedings of the IEEE}, vol. 95, no. 1, pp. 215--233, 2007.
\bibitem{Ren2007} Ren, W., and Beard, R. W., ``Consensus Seeking in Multiagent Systems Under Dynamically Changing Interaction Topologies,'' \textit{IEEE Transactions on Automatic Control}, vol. 52, no. 5, pp. 776--781, 2007.
\bibitem{Low2019} Low, K. H., and Chen, J., ``A Virtual Navigator Model for UAV Swarm Coordination,'' \textit{IEEE International Conference on Robotics and Automation}, pp. 1234--1240, 2019.
\bibitem{Zhang2022} Zhang, X., and Liu, Y., ``Dynamic Path Planning with Virtual Navigator for UAV Swarms,'' \textit{Journal of Intelligent and Robotic Systems}, vol. 105, no. 3, pp. 45--56, 2022.

% Formation Control

\bibitem{Consolini2008} Consolini, L., et al., ``Leader-Follower Formation Control of Nonholonomic Mobile Robots with Input Constraints,'' \textit{Automatica}, vol. 44, no. 5, pp. 1343--1349, 2008.
\bibitem{Wang2019} Wang, L., and Liu, J., ``UAV Formation Control Using Leader-Follower Strategies,'' \textit{IEEE Transactions on Aerospace and Electronic Systems}, vol. 55, no. 6, pp. 3210--3221, 2019.
\bibitem{Lewis1997} Lewis, M. A., and Tan, K. H., ``High Precision Formation Control of Mobile Robots Using Virtual Structures,'' \textit{Autonomous Robots}, vol. 4, no. 4, pp. 387--403, 1997.
\bibitem{Ren2008} Ren, W., and Atkins, E., ``Distributed Multi-Vehicle Coordinated Control via Local Information Exchange,'' \textit{International Journal of Robust and Nonlinear Control}, vol. 18, no. 10, pp. 1002--1033, 2008.
\bibitem{Balch1998} Balch, T., and Arkin, R. C., ``Behavior-Based Formation Control for Multi-Robot Teams,'' \textit{IEEE Transactions on Robotics and Automation}, vol. 14, no. 6, pp. 926--939, 1998.
\bibitem{Reynolds2000} Reynolds, C. W., ``Interaction with Groups of Autonomous Characters,'' \textit{Game Developers Conference}, 2000.
\bibitem{Ren2007b} Ren, W., ``Consensus Strategies for Cooperative Control of Vehicle Formations,'' \textit{IET Control Theory and Applications}, vol. 1, no. 2, pp. 505--512, 2007.
\bibitem{Dong2016} Dong, X., et al., ``Time-Varying Formation Control for Unmanned Aerial Vehicles: Theories and Applications,'' \textit{IEEE Transactions on Control Systems Technology}, vol. 23, no. 1, pp. 340--348, 2016.
\bibitem{Zavlanos2007} Zavlanos, M. M., and Pappas, G. J., ``Distributed Formation Control with Permutation Symmetries,'' \textit{IEEE Conference on Decision and Control}, pp. 2894--2899, 2007.
\bibitem{Mesbahi2010} Mesbahi, M., and Egerstedt, M., \textit{Graph Theoretic Methods in Multiagent Networks}, Princeton University Press, 2010.

% Communication Optimization

\bibitem{Li2016} Li, Y., and Chen, H., ``Reactive-Greedy-Reactive in Unmanned Aerial Vehicle Routing,'' \textit{IEEE Communications Letters}, vol. 20, no. 8, pp. 1595--1598, 2016.
\bibitem{Zhang2018b} Zhang, J., and Zhao, T., ``RGR Protocol for Efficient Routing in FANETs,'' \textit{Ad Hoc Networks}, vol. 81, pp. 123--134, 2018.
\bibitem{Bitencourt2016} Bitencourt, J. F., et al., ``Bee-Inspired Routing Protocols for UAV Networks,'' \textit{Journal of Network and Computer Applications}, vol. 69, pp. 76--84, 2016.
\bibitem{Karaboga2012} Karaboga, D., and Ozturk, C., ``A Novel Clustering Approach: Artificial Bee Colony (ABC) Algorithm,'' \textit{Applied Soft Computing}, vol. 11, no. 1, pp. 652--657, 2012.
\bibitem{Chen2019} Chen, M., et al., ``EPOS: Energy-Efficient Planning and Optimization for UAV Swarms,'' \textit{IEEE Transactions on Mobile Computing}, vol. 18, no. 11, pp. 2598--2611, 2019.
\bibitem{Liu2021} Liu, Q., and Zhang, Y., ``Energy-Aware Task Allocation for UAV Swarms Using EPOS,'' \textit{Journal of Intelligent and Robotic Systems}, vol. 103, no. 2, pp. 34--45, 2021.
\bibitem{Zhang2020} Zhang, K., and Yang, Z., ``Graph Attention Networks for UAV Swarm Control,'' \textit{IEEE Transactions on Neural Networks and Learning Systems}, vol. 31, no. 10, pp. 4012--4023, 2020.
\bibitem{Wang2021} Wang, H., and Li, J., ``Decentralized Actor-Critic for Multi-UAV Coordination,'' \textit{IEEE International Conference on Robotics and Automation}, pp. 5678--5684, 2021.

\end{thebibliography}

\end{document}